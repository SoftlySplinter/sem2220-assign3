\documentclass[11pt, a4paper]{article}

\usepackage{amsmath}
\usepackage{amssymb}
\usepackage{graphicx}
\usepackage{listings}
\usepackage{color}
\usepackage[section]{placeins}
\usepackage{paralist}
\usepackage{fullpage}
\usepackage{glossaries}
\usepackage{url}

\usepackage{caption}
\usepackage{subcaption}

\newcommand*{\titleGM}{\begingroup
\hbox{ 
\hspace*{0.2\textwidth} 
\rule{1pt}{\textheight} 
\hspace*{0.05\textwidth} 
\parbox[b]{0.75\textwidth}{ 

{\noindent\Huge\bfseries An Android Homepage Widget}\\[2\baselineskip] % Title
{\large \textit{SEM2220 Assignment 3}}\\[4\baselineskip] % Tagline or further description
{\Large \textsc{Alexander D Brown (adb9)}} % Author name

\vspace{0.5\textheight} 
}}
\endgroup}


\begin{document}
\titleGM 
\tableofcontents
\newpage

\section{Introduction}
This report details the process undertaken to produce an Android widget based 
on the existing code to load sessions from a SQLite database. This widget had 
several requirements, including the ability to select different days from the 
database as well as provide notifications read from a remote URL.

\section{Design}
The widget was designed in accordance with the Android App Widget Design 
Guidelines\cite{google2013widget}, which define guides for design constraints 
like the minimum and maximum size of the widget, layouts and backgrounds, etc.

To help conform to these guidelines, the template design 
pack\cite{google2013widgettemplates} provided by the Android Open Source 
Project under the Apache 2 License was used to design the widget. 

A mock version of the widget was created to view how it would appear on a 
device. From this is became obvious that the widget would need to be four 
cells wide (the maximum) by at least two cells tall.

This size would allow a view with the following information on it:

\begin{itemize}
  \item A title for the Widget
  \item Two buttons, one to move backwards through days and one to move forward
        through days
  \item A list of the sessions available for the specified day.
  \item The notification loaded from a remote site.
\end{itemize}

To keep the buttons accessible, they were made such that their minimum size was
a single cell each and surrounded the list of sessions to make the flow of data
natural. To conform to the iconography standards\cite{google2013iconography},
another resource was used from the Android Open Source Project; the Action Bar Icon Pack

The notification display was kept small so that it would not obstruct the view
of the data, but so that it would be easy to see at a glance. Finally, the 
title was given colour, based on the recommended 
colours\cite{google2013colour}, a purely aesthetic element.

\section{Development}


\section{Testing}


\section{Evaluation}


Breaking down the mark scheme the author has predicted the grade which should 
be given for each part, this is shown in table~\ref{tab:marks}.

Therefore, the author feels a mark of 82\% should be awarded. The values chosen 
were based on the following reasons:

\begin{description}
\item[Documentation] 
\item[Implementation] 
\item[Flair] 
\item[Testing] 
\end{description}

\begin{table}[h]
\centering
\begin{tabular}{|c|c|c|}\hline
\textbf{Part} & \textbf{Worth} & \textbf{Predicted Grade} \\ \hline
Documentation & 30\% & 25\% \\ 
Implementation & 50\% & 45\% \\ 
Flair & 10\% & 5\% \\ 
Testing & 10\% & 7\% \\ \hline
Total & 100\% & 82\% \\ \hline
\end{tabular}
\caption{Break Down of Marks}\label{tab:marks}
\end{table}

\newpage

\bibliographystyle{IEEEtran}
\bibliography{bibliography}

\end{document}
